\section{Questions \& Answers}
\subsection*{1. List some other popular Linux shells and describe their highlighted features}
\begin{enumerate}
    \item The C shell - denoted as \code{csh} or \code{tcsh} in \texttt{macOS} or \texttt{Red Hat Linux}
        \vspace{-0.5cm}
        \begin{itemize}
            \item Created by Bill Joy while he was a graduate student at University of California, Berkeley in the late 1970s.
            \item \textbf{Features:} C-like syntax, built-in arithmetic, command history, aliases definition, spelling correction...
        \end{itemize}
        \vspace{-0.5cm}
    \item The Z Shell - denoted as \code{zsh}
        \vspace{-0.5cm}
        \begin{itemize}
            \item Written by Paul Falstad in 1990 while he was a student at Princeton University. This is a Unix shell created as an extension for the Bourne shell.
            \item \textbf{Features:} Shared history among all running shell sessions, spelling corrections and command name autofill based on multiple factors: context-aware, Git branches, file paths...
        \end{itemize}
        \vspace{-0.5cm}
    \item Debian Almquist Shell - denoted as \code{dash}
        \vspace{-0.5cm}
        \begin{itemize}
            \item A Unix shell developed from the Almquist Shell \code{ash}, famous for being Debian and Ubuntu default shell
            \item \textbf{Features:} The shell is minimal and POSIX (Portable Operating System Interface) compliant, execution speeds up to 4x faster than bash and other shells, using less resource than other alternatives...
            \item However, \code{dash} is not \code{bash}-compatible, require additional reworkings of features not included in \code{dash} to run succesfully.
        \end{itemize}
        \vspace{-0.5cm}
\end{enumerate}
\subsection*{2. Compare the Output Redirection (\textgreater{} / \textgreater{}\textgreater{}) with the Piping (|) technique}
    \begin{tabular}{|>{\centering\arraybackslash}m{2.5cm}|>{\centering\arraybackslash}m{6cm}|>{\centering\arraybackslash}m{6cm}|}
        \hline
        \textbf{Aspects} & \makecell{\textbf{Output Redirection} \\(\textgreater{} / \textgreater{}\textgreater{})} & \makecell{\textbf{Piping}\\(|)}\\
        \hline
        Destination & File on disk & Another command\\
        \hline
        Result & Stored data & Processed data \\
        \hline
        Chaining & Single Operation & Multiple commands in sequence\\
        \hline
        Data flow & Command $\rightarrow$ file & Command $\rightarrow$ Command\\
        \hline
        Syntax & \makecell[l]{command \textgreater{} file (overwrite) \\ command \textgreater{}\textgreater{} file (append)} & command1 | command2\\
        \hline
    \end{tabular}
\subsection*{3. Compare the \texttt{sudo} and \texttt{su} commands}
    \begin{tabular}{|>{\centering\arraybackslash}m{3cm}|>{\centering\arraybackslash}m{6cm}|>{\centering\arraybackslash}m{6cm}|}
        \hline
        \textbf{Aspects} & \code{sudo} & \code{su}\\
        \hline
        Purpose & Executes a single command with privileges temporarily. & Switching from one user to another\\
        \hline
        Authentication &  Requires \texttt{current user's password}  & Requires \texttt{target user's password} \\
        \hline
        Session type & Doesn't change the current shell & Open a \texttt{new login shell} for the target user\\
        \hline
        Example & \code{sudo apt-get update} & \code{su} or \code{su -}\\
        \hline
    \end{tabular}
\subsection*{4. Discuss about the 777 permission on critical services (web hostings, sensitive databases,...)}
The 777 permission in Linux grants full read, write, and execute access to everyone — the owner, group, and others. Though this permission may seem convenient, it poses serious security risks: anyone can modify, delete, or execute the file, potentially 
leading to malware injection, privilege escalation, or data loss. Typically, 777 should only be used temporarily in testing environments, never in production systems. Instead, follow \textbf{"the principle of least privilege"}, use \code{chown} to assign the 
correct owner, apply specific permissions like \texttt{755} or \texttt{640}, and use the sticky bit (\texttt{1777}) for shared directories such as \code{/tmp}. Regular permission audits and a properly configured umask help ensure that the system remains both functional and secure.
\subsection*{5. What are the advantages of Makefile? Give examples}
A Makefile is a file used by the \code{make} build automation tool to control the compilation and building of programs - such as C/C++ and other compiled languages. It defines how to compile and link a program automatically, saving time and avoiding repetitive manual commands.\\
\quad \textbf{Advantages:}
\begin{itemize}
    \item \textbf{Automation:} Builds automatically based on dependencies
    \item \textbf{Efficiency:} Rebuilds only modified file
    \item \textbf{Consistency:}	Same build steps for everyone
    \item \textbf{Scalability:}	Handles large multi-file projects
    \item \textbf{Portability:}	Works across systems and environments
\end{itemize}
\quad \textbf{Example:}
\begin{lstlisting}
    CC = gcc
    CFLAGS = -Wall
    OBJ = main.o add.o sub.o
    TARGET = calculator

    $(TARGET): $(OBJ)
	    $(CC) $(CFLAGS) -o $(TARGET) $(OBJ)

    %.o: %.c
	    $(CC) $(CFLAGS) -c $<

    clean:
	    rm -f $(OBJ) $(TARGET)
\end{lstlisting}
\subsection*{6. Compiling a program in the first time usually takes a longer time in comparison with the next re-compiling. What is the reason?}
\quad First compilation compiles all source files from scratch. From the next compilation,
only the modified files are recompiled — the rest are reused from previous build. Makefile uses
file timestamps to detect which files are changed, so it saves time by avoiding unnecessary recompilation.

\subsection*{7. Is there any Makefile mechanism for other programming languages? If it has, give an example?}
\quad Yes, Makefile is not exclusive in C/C++. It can be used for any language or task that requires automation, e.g. \texttt{Python}, \texttt{JavaScript/TypeScript}, \texttt{PHP}...

	
